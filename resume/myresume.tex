%%%%%%%%%%%%%%%%%%%%%%%%%%%%%%%%%%%%%%%%%
% Medium Length Graduate Curriculum Vitae
% LaTeX Template
% Version 1.1 (9/12/12)
%
% This template has been downloaded from:
% http://www.LaTeXTemplates.com
%
% Original author:
% Rensselaer Polytechnic Institute (http://www.rpi.edu/dept/arc/training/latex/resumes/)
%
% Important note:
% This template requires the res.cls file to be in the same directory as the
% .tex file. The res.cls file provides the resume style used for structuring the
% document.
%
%%%%%%%%%%%%%%%%%%%%%%%%%%%%%%%%%%%%%%%%%

%----------------------------------------------------------------------------------------
%	PACKAGES AND OTHER DOCUMENT CONFIGURATIONS
%----------------------------------------------------------------------------------------

\documentclass[margin,line,10pt]{res} % Use the res.cls style, the font size can be changed to 11pt or 12pt here
\usepackage{amssymb}
\usepackage{enumitem}
\usepackage[none]{hyphenat}
\usepackage{anysize}

\marginsize{.6in}{2in}{.5in}{1.2in} %left,right,top,bottom



%\topmargin=-0.6in
%\oddsidemargin=-.5in %left side
%\evensidemargin=-.5in %right side
%\textwidth=6in

%\itemsep=0in
%\parsep=0in

\usepackage{helvet} % Default font is the helvetica postscript font
%\usepackage{newcent} % To change the default font to the new century schoolbook postscript font uncomment this line and comment the one above
\renewcommand{\familydefault}{\sfdefault}
\normalfont % this is required for the change to take effect

\usepackage{fancyhdr}
%\setlength{\textwidth}{5.3in} % Text width of the document
\renewcommand{\headrulewidth}{0pt} %remove horizontal rule
%\cfoot{\thepage\ of \pageref{LastPage}}
\cfoot{\hfill Page \thepage\ of 2 \hfill \textit{T. Chowdhury}}

\begin{document}
\pagestyle{fancy}


%----------------------------------------------------------------------------------------
%	NAME AND ADDRESS SECTION
%----------------------------------------------------------------------------------------

\moveleft.5\hoffset\centerline{\Large\bf Tasadduk Chowdhury} % Your name at the top
\vspace{8pt}

%\moveleft\hoffset\vbox{\hrule width\resumewidth height .7pt}\smallskip % Horizontal line after name; adjust line thickness by changing the '1pt'

%\moveleft.5\hoffset\centerline{Status: \textbf{US Citizen} \hfill 7215 Spring Cypress Rd, \#\ 821}
%\moveleft.5\hoffset\centerline{tchowdhury@uh.edu \hfill Spring, TX 77379}

\moveleft.5\hoffset\centerline{
tchowdhury@uh.edu \hfill   \hfill 
(713) 498-9788 \hfill  \hfill 
U.S. Citizen \hfill  \hfill 
Spring, TX 77379}



%----------------------------------------------------------------------------------------


\begin{resume}

%----------------------------------------------------------------------------------------
%	OBJECTIVE SECTION
%----------------------------------------------------------------------------------------

\section{OBJECTIVE}  
Seeking a Summer Research Intern position at Quantlab Financial.

\section{SUMMARY} 
\begin{itemize}[leftmargin=12pt] \itemsep -1pt 
\item PhD candidate in Applied Mathematics. Experienced in computed tomography, image and signal processing, numerical methods, and machine learning.
\item 7 years of programming experience in C/ C++ and MATLAB. Working knowledge of Unix/Linux operating systems. Strong background in object oriented programming and data structures.
\item Collaborated with a team of multidisciplinary researchers and solved a medical problem.
\item Demonstrated exceptional leadership skills by training students to write codes in MATLAB.
%\item Excellent verbal and written communication skills. 
\item Ability to give technical presentations at seminars and workshops.
%\item Outstanding presentation skills. Selected as the winner of the Graduate Student Research Presentation Competition in 2015.
\end{itemize}

%----------------------------------------------------------------------------------------
%	EDUCATION SECTION
%----------------------------------------------------------------------------------------

\section{EDUCATION}
{University of Houston, Houston, TX}
\\*[2pt]
{\bf Doctorate in Applied Mathematics}, GPA: 3.76/4.0 \hfill {\em Expected in May 2016} \\ 
Dissertation: Region-of-interest reconstruction algorithms in X-ray CT imaging. 
 \\*[5pt]
{\bf Master of Science in Applied Mathematics}, GPA: 3.72/4.0 \hfill {\em August 2012} 
 \\*[5pt]
{\bf Bachelor of Science in Mathematics}, Major GPA: 3.4/4.0  \hfill {\em August 2009} \\
Minor: \textbf{Computer Science}

\section{RELEVANT COURSES} 
Probability Theory, 
Statistical Analysis, 
Machine Learning, 
Stochastic Processes, 
Data Scientist's Toolbox,
R Programming,
Numerical Analysis, 
Mathematics in Medical Imaging,
Medical Imaging Physics, 
Differential Equations, 
Partial Differential Equations,
Wavelets and Compressed Sensing,
Data Structures, 
Scientific Computing, 
%Information Theory 
%Internet Computing, 
%Computer Graphics, 

\section{COMPUTER SKILLS} 

{\bf Programming:}  C/C++, MATLAB, Python, R, Shell Scripting, SQL \\
{\bf Operating System:} Unix/Linux, Mac OS, Windows \\
{\bf Publishing:}  HTML, CSS, \LaTeX \\
{\bf Software:} Excel, Visual Studio, GCC, Mathematica, Inkscape \\
{\bf Other:} Parallel Computing, Github

\section{PROJECT EXPERIENCE}
\textbf{Algorithms in Computed Tomography} \hfill {\em Fall 2012 - Spring 2015} \\
University of Houston, Houston, TX
\begin{itemize}[leftmargin=12pt] \itemsep -2pt
%\item Constructed numerical methods for accurate CT image reconstruction with novel geometries using localized X-ray scans for dose-reduction.
\item Derived a novel reconstruction method in X-ray CT to reduce the overall radiation exposure by localizing the X-rays and using wavelets based regularization.
\item Developed integrated codes in C/C++ and MATLAB using parallel computing techniques for both simulation of X-ray data acquisition and 3D image reconstruction.
\item Reduced 75\% of computation time by executing codes on a high performance computing cluster.
\end{itemize}

{\bf Automated Surgical Planning in Dentistry} \hfill {\em Summer 2012 - Spring 2013} \\
Houston Methodist \&\ University of Houston, Houston, TX
\begin{itemize}[leftmargin=12pt] \itemsep -2pt
\item Collaborated with a diverse team of mathematicians and medical scientists.
\item Developed an algorithm to assist doctors in surgical planning using principal component analysis (PCA) and mathematical optimization.
\item Implemented C++ and MATLAB, and validated on real data sets from patients.
\end{itemize}

\textbf{3D Object Recognition} \hfill {\em Spring 2014} \\
University of Houston, Houston, TX
\begin{itemize}[leftmargin=12pt] \itemsep -2pt
\item Implemented codes to solve the 3D object recognition problem using various machine learning methods including kernel PCA, SVM, KNN.
\item Verified algorithm using open image databases including COIL-100 and CALTECH.
\item Presented methods and results at the Machine Learning Workshop on campus.
\end{itemize}

\newpage
{\bf Digital Image Processing} (Master's Research)\hfill {\em Fall 2011 - Spring 2012} \\
University of Houston, Houston, TX
\begin{itemize}[leftmargin=12pt] \itemsep -2pt
\item Studied image processing methods based on mathematical morphology.
\item Implemented algorithms for noise removal, edge detection, and pattern matching.
\item Developed codes in C and MATLAB, and tested on a large set of 2D and 3D images.
\end{itemize}

\section{LEADERSHIP EXPERIENCE}

{\bf Graduate Teaching Assistant \& Tutor} \hfill {\em Fall 2011 - Present} \\
University of Houston, Houston, TX
\begin{itemize}[leftmargin=12pt] \itemsep -2pt % Reduce space between items
\item Conduct calculus recitations in a classroom of 60 students.
\item Grade homework and exams for math courses including calculus, probability and statistics, differential equations, and complex analysis.
\item Provide one-on-one tutoring to students at the math tutoring center on campus.
\end{itemize}

{\bf Instructor of Linear Algebra} \hfill {\em Fall 2014, Fall 2015} \\
University of Houston, Houston, TX
\begin{itemize}[leftmargin=12pt] \itemsep -2pt
\item Teach junior level linear algebra for a class of 70 students.
\item Design course syllabus and prepare homework, quizzes, and exams.
\item Train students to use MATLAB to solve numerical problems.
\end{itemize}


%\section{WORK EXPERIENCE}

%{\bf Shift Manager} \hfill {\em July 2004 - May 2011} \\
%Jack In The Box, Houston, TX

%\begin{itemize} \itemsep -2pt % Reduce space between items
%\item Provided customer service, created schedules, and helped in hiring of new employees.
%\item Trained new employees effectively on food safety, cash handling procedures, and point of sales system.
%\item Completed training on food safety and received certificate of achievements from the National Restaurant Association Educational Foundation.
%\end{itemize}
 
%\section{PRESENTATIONS}

\section{AWARDS \& FELLOWSHIPS} 
Student Travel Grant - 13th Fully 3D Meeting, {\em June 2015} \\
Best Graduate Student Research Presentation Award, {\em May 2015} \\
Doctoral Student Tuition Fellowship, {\em August 2013} \\
Graduate Assistant Teaching Fellowship, {\em August 2011}

\section{PUBLICATIONS}
``An iterative algorithm for region-of-interest reconstruction with cone-beam acquisitions on a generic source trajectory'', \textbf{T. Chowdhury}, A. Sen, R. Azencott. Proc of 13th Fully 3D Meeting, 2015.

``Region-of-interest reconstructions from truncated 3D x-ray projections'', R. Azencott, B. G. Bodmann, \textbf{T. Chowdhury}, D. Labate, A. Sen, and D. Vera. (to be submitted).


\section{ACTIVITIES} 
{\bf President,} UH SIAM Student Chapter \hfill {\em Fall 2015 - Present} \\
{\bf Vice President,} UH SIAM Student Chapter \hfill {\em Fall 2014 - Spring 2015} \\
{\bf Webmaster,} UH SIAM Student Chapter \hfill {\em Fall 2014 - Present} \\
Imaging Research Meeting, University of Houston \hfill {\em Fall 2012} \\
Society for Industrial and Applied Mathematics (SIAM) \hfill {\em Fall 2011 - Present} \\
American Mathematical Society (AMS) \hfill {\em Fall 2011 - Present}

%----------------------------------------------------------------------------------------

\end{resume}
\end{document}